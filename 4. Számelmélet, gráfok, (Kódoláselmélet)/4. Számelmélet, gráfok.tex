\documentclass[margin=0px]{article}

\usepackage{listings}
\usepackage[utf8]{inputenc}
\usepackage{graphicx}
\usepackage{float}
\usepackage[a4paper, margin=0.7in]{geometry}
\usepackage{subcaption}
\usepackage{amsthm}
\usepackage{amssymb}
\usepackage{amsmath}
\usepackage{fancyhdr}
\usepackage{setspace}

\onehalfspacing

\renewcommand{\figurename}{ábra}
\newenvironment{tetel}[1]{\paragraph{#1 \\}}{}

\newcommand{\N}{\mathbb{N}}
\newcommand{\Z}{\mathbb{Z}}
\newcommand{\R}{\mathbb{R}}
\newcommand{\Q}{\mathbb{Q}}
\newcommand{\C}{\mathbb{C}}

\pagestyle{fancy}
\lhead{\it{PTI BSc Záróvizsga tételek}}
\rhead{4. Számelmélet, gráfok}

\title{\textbf{{\Large ELTE IK - Programtervező Informatikus BSc} \vspace{0.2cm} \\ {\huge Záróvizsga tételek}} \vspace{0.3cm} \\ 4. Számelmélet, gráfok}
\author{}
\date{}

\begin{document}
\maketitle

\begin{tetel}{Számelmélet, gráfok, kódoláselmélet}
    Halmazok, relációk, függvények és műveletek. Komplex számok. Leszámlálások véges halmazokon. Irányítatlan és irányított gráfok, fák, Euler-és Hamilton-gráfok, gráfok adatszerkezetei. Számelméleti alapfogalmak, oszthatóság, kongruencia, prímek. Polinomok és műveleteik, maradékos osztás.
\end{tetel}
\section{Számelmélet}
\subsection{Halmazok}
A halmaz (rendszer, osztály, összesség, ...) elemeinek gondolati burka. Egy halmazt az elemei egyértelműen meghatároznak.
\begin{description}
    \item[Alapfogalmak] \hfill
        \begin{itemize}
            \item Üres halmaz \\
                  Az a halmaz, amelynek nincs eleme az Üres halmaz. Jele: $\emptyset$ . A meghatározottsági axióma alapján ez egyértelmű
            \item Részhalmaz \\
                  Azt mondjuk, hogy \textbf{A részhalmaza B-nek} ($A \subseteq B$), ha $\forall a \in A : a \in B$, azaz A minden elemét tartalmazza B.
                  \textbf{A valódi részhalmaza B-nek} \(A \subset B\), ha $A \subseteq B$, de $A \neq B$, azaz B-nek van legalább egy olyan eleme, ami nem eleme A-nak.
            \item Hatvány halmaz \\
                  Ha {\it A} egy halmaz, akkor azt a halmazrendszert, melynek elemei pontosan az {\it A} halmaz részhalmazai az {\it A} hatványhalmazának mondjuk, és $2^A$-val jelöljük.
                  \begin{itemize}
                      \item $A = \emptyset , 2^\emptyset = {\emptyset}$
                      \item $A = \{a\} , 2^{\{a\}} = \{\emptyset, \{a\}\}$
                      \item $A = \{a, b\} , 2^{\{a,b\}} = \{\emptyset, \{a\}, \{b\}, \{a, b\}\}$
                      \item $|2^A| = 2^{|A|}$
                  \end{itemize}
        \end{itemize}
    \item[Műveletek] \hfill
        \begin{itemize}
            \item Unió \\
                  Az {\it A} és {\it B} halmazok uniója: $A \cup B$ az a halmaz, mely pontosan az {\it A} és a {\it B} elemeit tartalmazza.
            \item Metszet \\
                  Az {\it A} és {\it B} halmazok metszete: $A \cap B$ az a halmaz, mely pontosan az {\it A} és a {\it B} közös elemeit tartalmazza: $A \cap B = \{ x \in A : x \in B \}$
            \item Diszjunkt \\
                  Ha $A \cap B = \emptyset$, akkor {\it A} és {\it B} diszjunktak.
            \item Különbség \\
                  Az {\it A} és {\it B} halmazok különbsége az $A \diagdown B = \{ x \in A : x \notin B \}$
            \item Komplementer \\
                  Egy rögzített {\it X} alaphalmaz és $A \subseteq X$ részhalmaz esetén az {\it A} halmaz komplementere az $ \overline{A} = A' = X \diagdown A$
            \item Szimmetrikus differencia \\
                  $A \triangle B = (A \diagdown B) \cup (B \diagdown A)$
        \end{itemize}
    \item[Tulajdonságok] \hfill
        \begin{itemize}
            \item Unió
                  \begin{itemize}
                      \item $A \cup \emptyset = A$
                      \item $A \cup ( B \cup C ) = (A \cup B) \cup C$ (asszociativitás)
                      \item $A \cup B = B \cup A$ (kommutativitás)
                      \item $A \cup A = A$ (idempotencia)
                      \item $A \subseteq B \Longleftrightarrow A \cup B = B$
                  \end{itemize}
            \item Metszet
                  \begin{itemize}
                      \item $A \cap \emptyset = \emptyset$
                      \item $A \cap ( B \cap C ) = (A \cap B) \cap C$ (asszociativitás)
                      \item $A \cap B = B \cap A$ (kommutativitás)
                      \item $A \cap A = A$ (idempotencia)
                      \item $A \subseteq B \Longleftrightarrow A \cap B = A$
                  \end{itemize}
            \item Unió és Metszet disztributivitási tulajdonságai
                  \begin{itemize}
                      \item $A \cap ( B \cup C ) = (A \cap B) \cup (A \cap C)$
                      \item $A \cup ( B \cap C ) = (A \cup B) \cap (A \cup C)$
                  \end{itemize}
            \item Különbség
                  \begin{itemize}
                      \item $A \diagdown B = A \cap \overline{B}$
                  \end{itemize}
            \item Komplementer
                  \begin{itemize}
                      \item $\overline{\overline{B}} = A$
                      \item $\overline{\emptyset} = X$
                      \item $\overline{X} = \emptyset$
                      \item $A \cap \overline{A} = \emptyset$
                      \item $A \cup \overline{A} = X$
                      \item $A \subseteq B \Longleftrightarrow \overline{B} \subseteq \overline{A}$
                      \item $\overline{A \cap B} = \overline{A} \cup \overline{B}$
                      \item $\overline{A \cup B} = \overline{A} \cap \overline{B}$
                  \end{itemize}
            \item Szimmetrikus differencia
                  \begin{itemize}
                      \item $A \triangle B = (A \cup B) \diagdown (B \cap A)$
                  \end{itemize}
        \end{itemize}
\end{description}

\subsection{Relációk, rendezések}
\begin{description}
    \item[Alapfogalmak] \hfill
        \begin{itemize}
            \item Rendezett pár \\
                  $(x,y)$ rendezett pár, ha $(x,y) = (u,v) \Longleftrightarrow x = u \ \land \ y = v$. Ezt a tulajdonságot halmazokkal definiáljuk:
                  \[ (x,y) := \{ \{x\}, \{x, y\} \} \]
            \item Descartes-szorzat \\
                  $X,Y$ halmazok Descartes-szorzata vagy direkt szorzata:
                  \[ X \times Y := \{ (x,y) : x \in X, y \in Y \} \]
            \item Binér reláció \\
                  Egy halmazt binér relációnak nevezünk, ha minden eleme rendezett pár.
                  Ha $R$ binér reláció és $(x,y) \in R$, akkor gyakran írjuk: $xRy$
            \item Reláció \\
                  Ha $X,Y$ halmazokra $R \subset X\times Y$, akkor $R$ reláció $X$ és $Y$ között.
            \item Értelmezési tartomány \\
                  Az $R$ binér reláció értelmezési tartománya:
                  \[ \textrm{dmn}(R) := \{ x\  | \ \exists y : (x,y)\in R  \}\]
            \item Érték készlet \\
                  Az $R$ binér reláció érték készlete:
                  \[ \textrm{rng}(R) := \{ y\  | \ \exists x : (x,y)\in R  \}\]
            \item Inverz \\
                  Egy $R$ binér reláció inverze:
                  \[R^{-1} := \{(a,b) : (b,a) \in R\} \]
            \item Halmaz képe \\
                  Legyen $R$ binér reláció, és $A$ halmaz. Az $A$ halmaz képe:
                  \[R(A) := \{y \ | \ \exists x\in A: (x,y) \in R\} \]
            \item Kompozíció \\
                  $R$ és $S$ binér relációk kompozíciója:
                  \[ R \circ S := \{ (x,y) \ | \ \exists z : (x,z) \in S \ \land \ (z,y) \in R \ \} \]
        \end{itemize}
    \item[Tulajdonságok] \hfill \\
        Az $R$ egy $X$-beli binér reláció (azaz $R \subset X\times X$)
        \begin{enumerate}
            \item tranzitív \\
                  \[\forall x,y,z : (x,y)\in R \ \land \ (y,z) \in R \Longrightarrow (x,z) \in R \]
            \item szimmetrikus \\
                  \[\forall x,y : (x,y)\in R \Longrightarrow (y,x) \in R \]
            \item antiszimmetrikus \\
                  \[\forall x,y : (x,y)\in R \ \land \ (y,x) \in R \Longrightarrow x = y\]
            \item szigorúan antiszimmetrikus \\
                  \[\forall x,y : (x,y)\in R \Longrightarrow (y,x) \notin R\]
            \item reflexív \\
                  \[\forall x \in X : (x,x)\in R\]
            \item irreflexív \\
                  \[\forall x \in X : (x,x)\notin R\]
            \item trichotóm \\
                  Ha minden $x,y \in X$ esetén az alábbiak közül pontosan egy teljesül
                  \begin{enumerate}
                      \item[a)] $x=y$
                      \item[b)] $(x,y) \in R$
                      \item[c)] $(y,x) \in R$
                  \end{enumerate}
            \item dichotóm \\
                  \[\forall x,y \in X : (x,y) \in R \ \lor \ (y,x) \in R\]
                  Más néven az elemek összehasonlíthatóak.
        \end{enumerate}
    \item[Rendezések] \hfill
        \begin{itemize}
            \item Ekvivalenciareláció, osztályozás \\
                  $X$ halmaz, $R$ $X$-beli binér reláció ekvivalenciareláció, ha
                  \begin{itemize}
                      \item Reflexív
                      \item Tranzitív
                      \item Szimmetrikus
                  \end{itemize}

                  $X$ részhalmazainak egy $\mathcal{O}$ rendszerét osztályozásnak hívjuk, ha $\mathcal{O}$ páronként diszjunkt nemüres halmazokból álló halmazrendszer, melyre $\cup\mathcal{O} = X$
                  \\\\
                  Tétel: \\
                  Egy ekvivalenciareláció meghatároz egy osztályozást. Fordítva: $\mathcal{O}$ osztályozásra \\ ${R = \cup\{Y\times Y : Y \in \mathcal{O} \}}$ ekvivalenciareláció.

            \item Részbenrendezés\\
                  $X$ halmaz, $R$ $X$-beli binér reláció részbenrendezés, ha
                  \begin{itemize}
                      \item Reflexív
                      \item Tranzitív
                      \item Antiszimmetrikus
                  \end{itemize}
            \item Teljes rendezés \\
                  $X$ halmaz, $R$ $X$-beli binér reláció (teljes) rendezés, ha
                  \begin{itemize}
                      \item Reflexív
                      \item Tranzitív
                      \item Antiszimmetrikus
                      \item Dichotóm
                  \end{itemize}
                  Magyarul ha egy részbenrendezés dichotóm (tehát minden eleme összehasonlítható), akkor (teljes) rendezés.
            \item Szigorú és gyenge reláció, rendezés \\
                  $X$ halmaz, $R$,$S$ relációk $X$-beliek. Ha
                  \[ xRy \ \land \ x \neq y \Rightarrow xSy\]
                  akkor $S$-et az $R$ szigorításának nevezzük.\\
                  Megfordítva, ha
                  \[  xRy \ \lor \ x = y \Rightarrow xTy \]
                  akkor $T$ az $R$-hez megfelelő gyenge reláció.\\

                  \textit{Megjegyzés: Tulajdonképpen a reflexívitás elvételéről és hozzáadásáról van szó. Egy részbenrendezés esetén a megfelelő szigorú reláció (szigorú részbenrendezés) tehát irreflexív, következésképpen szigorúan antiszimmetrikus is. Megfordítva: Egy X-beli szigorú részbenrendezés (tran., irrefl., szig. ant.) megfelelő gyenge relációja részbenrendezés. }
        \end{itemize}
    \item[Korlátok] \hfill
        \begin{itemize}
            \item Legkisebb, legnagyobb, minimális, maximális elem \\
                  $X$ halmazbeli részbenrendezés ($\preccurlyeq$) legkisebb (legelső) elemén egy olyan $x\in X$ elemet értünk, melyre: $\forall y \in X : x\preccurlyeq y$. (Ilyen nem biztos, hogy létezik, de ha igen, akkor egyértelmű).\\
                  Hasonlóan a legnagyobb (utolsó) elem olyan $x\in X$, hogy $\forall y \in X : y \preccurlyeq x$.\\

                  $x$-et minimálisnak nevezzük, ha nincs nála kisebb elem, maximálisnak, ha nincs nála nagyobb elem. (Szemben a legkisebb/legnagyobb elemekkel, minimális/maximális elemből több is lehet. Ha viszont $X$ rendezett, akkor legkisebb=minimális, legnagyobb=maximális.)
            \item Alsó, felső korlát \\
                  $X$ részbenrendezett halmaz, $Y \subset X$. Az $x \in X$ elem az $Y$ alsó korlátja $\forall y \in Y : x \preccurlyeq y$. (felső korlátja: $\forall y \in Y : y \preccurlyeq x$). Látható, hogy $x$ nem feltétlenül eleme $Y$-nak, sőt az is lehet, hogy $Y$-nak nincs alsó/felső korlátja, vagy akár több is van. Ha azonban $x\in Y$, akkor egyértelmű és ez Y legkisebb eleme.
            \item Infimum, szuprémum \\
                  Ha az alsó korlátok között van legnagyobb elem, azt $Y$ alsó határának, infimumának nevezzük. (Jele: inf$Y$) \\
                  Ha a felső korlátok között van legnagyobb elem, azt $Y$ felső határának, szuprémumának nevezzük. (Jele: sup$Y$)
            \item Alsó, felső határ tulajdonság \\
                  $X$ részbenrendezett halmaz. Ha $ \forall \emptyset \neq Y \subset X$ : $Y$ felülről korlátos és van szuprémuma, akkor felső határ tulajdonságú.
                  Illetve ha $ \forall \emptyset \neq Y \subset X$ : $Y$ alulról korlátos és van infimuma, akkor alsó határ tulajdonságú.
        \end{itemize}
\end{description}
\subsection{Függvények és műveletek}
\subsubsection{Függvények}
\begin{description}
    \item[Definíció] \hfill \\
        Egy $f$ reláció függvény, ha
        \[ (x,y) \in f \ \land (x,y') \ \in f \Longrightarrow y = y' \]
        Más szóval minden $x$-hez legfeljebb egy olyan $y$ létezik, hogy $(x,y) \in f$

        Így minden $x \in$ dmn($f$)-re az $f(x) = \{y\}$, melyet $f(x) = y$ vagy $f: x \mapsto y$ vagy $f_x = y$ is szoktunk jelölni.

    \item[Értelmezési tartomány, értékkészlet] \hfill \\
        Az $f : X \rightarrow Y$ jelölést használjuk, ha dmn($f$) $ = X $. \\
        Az $f \in X \rightarrow Y$ jelölést használjuk, ha dmn($f$) $\subset X$ (amikor dmn($f$)$ \subsetneq X$ is előfordulhat).\\
        Mindkét esetben rng($f$) $\subset Y$.

    \item[Injektív] \hfill \\
        $f$ függvény kölcsönösen egyértelmű/injektív, ha
        \[ f(x) = y \ \land \ f(x') = y \ \Longrightarrow \ x = x' \]
        Ez azzal ekvivalens, hogy $f^{-1}$ reláció is függvény.
    \item[Szürjektív] \hfill \\
        Az $f$ függvény szürjektív, ha
        \[ \forall y \in Y : \exists x\in X : f(x) = y\]
        Azaz rng($f$) = $Y$. Magyarul az $f$ függvény az egész $Y$-ra képez.
    \item[Bijektív] \hfill \\
        Ha az $f$ függvény injektív és szürjektív, akkor bijektív.
    \item[Indexelt család] \hfill \\
        Az $x$ függvény $i$ helyen felvett értékét $x_i$-vel is szoktuk jelölni. Ilyenkor gyakran dmn($f$) = $I$ értelmezési tartományt indexhalmaznak, elemeit indexeknek, rng($f$)-et indexelt halmaznak, és magát az $x$ függvényt indexelt családnak szoktuk nevezni.
\end{description}
\subsubsection{Műveletek}
\begin{description}
    \item[Definíciók] \hfill
        \begin{itemize}
            \item Binér művelet \\
                  $X$ halmazon egy $f : X \times X \rightarrow X$ függvény binér művelet.
            \item Unér művelet \\
                  $X$ halmazon egy $f : X \rightarrow X$ függvény unér művelet.
            \item Nullér művelet \\
                  $X$ halmaz, $f : \{\emptyset \} \rightarrow X $ nullér művelet. (Gyakorlatilag elemkiválasztás)
        \end{itemize}
    \item[Tulajdonságok] \hfill
        \begin{itemize}
            \item Legyen $\spadesuit, \copyright$ binér műveletek $X$-en.
                  \begin{enumerate}
                      \item $\spadesuit$ asszociatív, ha
                            \[ \forall x,y,z \in X : \quad (x \ \spadesuit \ y ) \ \spadesuit \ z = x \  \spadesuit \  (y \ \spadesuit z) \]

                      \item $\spadesuit$ kommutatív, ha
                            \[ \forall x,y \in X : \quad x \ \spadesuit \ y = y \  \spadesuit \  x \]

                      \item $\spadesuit$ disztributív a $\copyright$-ra, ha $\forall x,y,z \in X$:
                            \[ x \  \spadesuit \  (y \ \copyright \ z) = (x \  \spadesuit \ y) \ \copyright \ (x \ \spadesuit \ z) \quad \textrm{ - baloldali} \]
                            \[ (y \ \copyright \ z) \  \spadesuit \ x = (y \  \spadesuit \ x) \ \copyright \ (z \ \spadesuit \ x) \quad \textrm{ - jobboldali} \]
                  \end{enumerate}

            \item Legyen $\heartsuit$ binér művelet $X$-en és $\S$ binér művelet $Y$-on
                  $f : X \rightarrow Y$ művelettartó ha:
                  \[ \forall x_1,x_2 \in X:  f(x_1 \heartsuit x_2) = f(x_1) \ \S \ f(x_2) \]
        \end{itemize}
\end{description}
\subsection{Számfogalom, komplex számok}
\subsubsection{Számfogalom}
\begin{description}
    \item[Algebrai Struktúrák] \hfill
        \begin{enumerate}
            \item Grupoid \\
                  $G$ halmaz egy $\star$ művelettel, azaz a $(G, \star)$ párt grupoidnak nevezzük.
            \item Félcsoport\\
                  Ha egy grupoidban a $\star$ művelet asszociatív, akkor a grupoid félcsoport.
            \item Monoid \\
                  Semleges elemes félcsoportot monoidnak nevezzük.\\
                  \textit{Megyjegyzés: $ a \in G$ semeleges elem, ha $\forall g \in G : a \star g = g \star a = g$}
            \item Csoport \\
                  Ha egy monoidban minden elemnek van inverze, akkor csoportról beszélünk. \\
                  \textit{Megyjegyzés: $g,g^{-1} \in G$ és $\xi\in G$ semleges elem, akkor a $g^{-1}$ a $g$ inverze, ha $g\star g^{-1} = \xi$ és $g^{-1} \star g = \xi$}
            \item Ábel-csoport \\
                  Ha egy csoportban a művelet kommutatív, akkor Abel-csoport.
            \item Gyűrű \\
                  $(R,+,\cdot )$ gyűrű, ha az összeadással Abel-csoport, a szorzással félcsoport és teljesül mindkét oldali disztributivitás.

                  Ha a szorzás kommutatív, akkor kommutatív gyűrű.

                  Ha a szorzásnak van egységeleme, akkor egységelemes gyűrű.
            \item Integritási tartomány \\
                  Nullosztó mentes kommutatív gyűrű.

                  \textit{Nullosztó: $x,y$ nullátók különböző elemek, de $x\cdot y = 0$}
            \item Rendezett integritási tartomány \\
                  $R$ integritási tartomány rendezett integritási tartomány, ha rendezett halmaz, továbbá az összeadás és szorzás monoton.

                  \textit{Összeadás monoton: $x,y,z \in R$ és $x \leq y \ \Rightarrow \ x+z \leq y+z$ \\
                      Szorzás monoton: $x,y \in R$ és $x,y\geq0 \ \Rightarrow \ x\cdot y \geq 0$ }
            \item Test \\
                  Egy $R$ gyűrűt, ha $R\setminus\{0\}$ szorzással Abel-csoport, akkor test.
            \item Rendezett test \\
                  Ha egy test rendezett integritási tartomány, akkor rendezett test.
        \end{enumerate}
        \item[Természetes számok]\hfill
        \begin{itemize}
            \item Peano-axiómák \\
                  Legyen $\N$ egy halmaz és a $^+$ egy $\N$-en értelmezett függvény. Az alábbi feltételeket Peano-axiómáknak nevezzük:
                  \begin{enumerate}
                      \item $0 \in \N \qquad$ - $0$ egy nullér művelet $\N$-en
                      \item ha $n \in \N$, akkor $n^+ \in N \qquad$ - $^+$ egy unér művelet $\N$-en
                      \item ha $n \in \N$, akkor $n^+ \neq 0 \qquad $ - $0$ nincs a $^+$ értékkészletében
                      \item ha $n,m \in \N$, és $m^+ = n^+$, akkor $n = m \qquad $ - $^+$ injektív
                      \item ha $S \subset \N, 0 \in S$, továbbá $ n \in S : n^+\in S$, akkor $S = \N \qquad $ - a matematikai indukció elve
                  \end{enumerate}
            \item Műveletek
                  \begin{itemize}
                      \item összeadás \\
                            $k,m,n \in \N$, akkor:
                            \begin{enumerate}
                                \item $(k+m)+n = k+(m+n)$ \textit{- asszociativitás}
                                \item $n+0 = 0+n = n$ \textit{- 0 a nullelem (additív semleges elem)}
                                \item $n+k = k+n$ \textit{- kommutativitás}
                                \item $n+k = m+k$ vagy $k+n = k+m$, akkor $m=n$ \textit{- egyszerűsítési szabály}
                            \end{enumerate}
                      \item szorzás \\
                            $k,m,n \in \N$, akkor:
                            \begin{enumerate}
                                \item $(k\cdot m)\cdot n = k\cdot (m\cdot n)$ \textit{- asszociativitás}
                                \item $ 0\cdot n = n\cdot 0 = 0$
                                \item $n\cdot 1 = 1\cdot n = n$ \textit{- 1 az egységelem (multiplikatív semleges elem)}
                                \item $n\cdot k = k\cdot n$ \textit{- kommutativitás}
                                \item $k\cdot (m+n) = k\cdot m + \cdot n$, illetve $(m+n) \cdot k = m\cdot k+n\cdot k$ \textit{- disztributivitás}
                                \item $k\neq 0$ esetén: $n\cdot k = m\cdot k$, akkor $m=n$ \textit{- egyszerűsítési szabály}
                            \end{enumerate}
                  \end{itemize}
        \end{itemize}
    \item[Egész számok] \hfill \\
        Természetes számok körében az összeadásra nézve csak a nullának van inverze, másként szólva, a kivonás általában nem végezhető el.

        Tekintsük a $\sim\ \subset\N\times\N$ relációt, melyre $(m,n) \sim (m',n')$, ha $m+n' = m'+n$. És vegyük az $(m,n)+(m',n') = (m+m',n+n')$ összeadást. A $\sim$ reláció ekvivalenciareláció, az ekvivalenciaosztályok halmazát jelöljük $\Z$-vel. $\Z$ elemeit egész számoknak nevezzük.

        Az összeadás kompatibilis az ekvivalenciával, így az egész számok között értelmezve van, és $(\Z, +)$ Ábel-csoport.

        Tehát $(\Z, +, \cdot)$ gyűrű.

        \textit{Megjegyzés: $*$ művelet kompatibilis a $\asymp$ ekvivalenciarelációval, ha teljesül: $ x \asymp x' \ \land \ y \asymp y' \ \Longrightarrow \ x * y \asymp x'*y'$}
    \item[Racionális számok] \hfill \\
        Az egész számok körében a nem nulla elemek közül csak az 1-nek és a $-1$-nek van multiplikatív inverze, másként szólva az osztás általában nem végezhető el.

        Tekintsük a $\Z \times (\Z\setminus\{0\})$-n a $\sim$ relációt, melyre $(m,n) \sim (m',n')$, ha $mn' = nm'$. És vegyük az $(m,n)+(m',n') = (mn'+nm', nn')$ összeadást és az $(m,n)\cdot(m',n')=(mm', nn')$ szorzást. A $\sim$ reláció ekvivalenciareláció, az ekvivalenciaosztályok halmazát jelöljük $\Q$-val. $\Q$ elemeit racionális számoknak nevezzük.

        $(\Q, +, \cdot)$ rendezett test.
    \item[Valós számok] \hfill \\
        Nincs olyan $a \in \Q$ szám, melynek négyzete 2. Tehát nem minden szám írható fel m/n ($m,n \in \N^+$) alakban.

        Archimédeszi rendezettség:\\
        Egy $F$ rendezett testet archimédeszien rendezett, ha $x,y\in F: \exists n \in \N : nx \geq y \quad (x>0)$

        A racionális számok rendezett teste archimédeszien rendezett, de nem felső határ tulajdonságú.

        Egy felső határ tulajdonságú rendezett testet a valós számok testének nevezünk, és $\R$-rel jelöljük. ($\exists!\R$)
\end{description}
\subsubsection{Komplex számok}
A komplex számok szükségét a harmadfokú egyenletek megoldására való Cardano-képlet szülte. Ugyanis abban az esetben, amikor az egyenletnek három különböző valós gyöke van, a képletben a gyökjel alá negatív szám kerül. Fokozatosan tisztult a "képzetes" számokkal való számolás szabályai, és a trigonometrikus függvényekkel való kapcsolat.

\begin{description}
    \item[Definíció] \hfill \\
        A komplex számok halmaza $\C = \R \times \R$. $\C$ az $(x,y)+(x',y') = (x+x', y+y')$ összeadással és az $(x,y)\cdot(x',y') = (xx'-yy', y'x+yx')$ szorzással test. A komplex számok halmaza nem rendezett test, mivel (tétel alapján) egy rendezett integritási tartományban $ x \neq 0 \ \Rightarrow x^2 > 0$. (Ez azonban $(0,1)^2=i^2 = -1$-re nem teljesül).

            [A komplex számok körében (0,0) a nullelem, $(1,0)$ egységelem, $(x,y)$ additív inverze $(-x,-y)$, és $(0,0) \neq (x,y)$ pár multiplikatív inverze az $(\frac{x}{x^2+y^2}, \frac{-y}{x^2+y^2})$ pár.]
    \item[Valós számok azonosítása] \hfill \\
        Mivel $(x,0)+(x',0) = (x+x',0)$ és $(x,0)\cdot(x',0) = (xx',0)$ így az összes $(x,0), x\in\R$ komplex számot azonosíthatjuk $\R$-rel.
    \item[Komplex számok algebrai alakja] \hfill \\
        Mivel
        \[(x,y) = (x,0)+(y,0)\cdot i = x+yi\]
        így a komplex számokat $a+bi$ algebrai alakban is írhatjuk.

        Ekkor az Re($z$) = $x$ valós számot a $z = (x,y)$ komplex szám valós részének, az Im($z$) = $y$ valós számot pedig a képzetes  részének nevezzük.
    \item[Konjugált] \hfill \\
        $ z = x+yi$ komplex szám konjugáltja: $\overline{z} = x-yi$

        Tulajdonságai:
        \begin{enumerate}
            \item $\overline{z+w} = \overline{z}+\overline{w}$
            \item $\overline{z\cdot w} = \overline{z}\cdot\overline{w}$
            \item $\overline{\overline{z}} = z$
            \item $z + \overline{z}$ = 2Re($z$)
            \item $z - \overline{z}$ = $i\cdot2$Im($z$)
        \end{enumerate}
    \item[Abszolút érték] \hfill \\
        A $z=(x,y)$ komplex szám abszolút értéke: $|z| = \sqrt{x^2+y^2}$

        Tulajdonságai:
        \begin{enumerate}
            \item $z\cdot\overline{z} = {|z|}^2$
            \item $\frac{1}{z}= \frac{\overline{z}}{|z|^2} $
            \item $|z| = \overline{|z|}$
            \item $|z\cdot w | = |z|\cdot|w|$
            \item $|z+w| \leq |z| +|w|$
        \end{enumerate}
    \item[Trigonometrikus alak] \hfill
        \begin{itemize}
            \item Argumentum \\
                  $z \neq 0$ esetén az a $z$ argumentuma $\forall t \in\R$, melyre Re($z$) = $|z|$cos($t$), és Im($z$) = $|z|$sin($t$). Más szóval a $z$ argumentuma az origóból a $z$-be mutató vektor és a pozitív valós tengellyel bezárt szöge.
            \item Trigonometrikus alak \\
                  A $z$ komplex szám trigonometrikus alakja: $ z = |z|($cos($t$)+$i\cdot$sin($t$)
            \item Moivre-azonosságok \\
                  Legyen $z = |z|($cos($t$)+$i\cdot$sin($t$)), és $w = |w|($cos($s$)+$i\cdot$sin($s$)). Ekkor
                  \[z\cdot w = |z||w|(\textrm{cos}(t+s)+i\cdot \textrm{sin}(t+s))\]
                  \[\frac{z}{w} = \frac{|z|}{|w|}(\textrm{cos}(t-s)+i\cdot \textrm{sin}(t-s)) \quad (w \neq 0)\]
                  \[ z^n = |z|^n(\textrm{cos}(nt)+i\cdot \textrm{sin}(nt)) \quad (n \in \Z)\]
            \item Gyökvonás \\
                  Legyen $z^n = w$ ekkor:
                  \[ \sqrt[n]{w} = \Bigg\{z_k = \sqrt[n]{|w|}\bigg(\textrm{cos}\Big(\frac{t+2k\pi}{n}\Big)+\textrm{sin}\Big(\frac{t+2k\pi}{n}\Big)\bigg), k=0,...,n-1\Bigg\} \]
                  De mivel ez a jelöltés összetéveszthető a valósak között (egyértelművé tett) valós gyökvonással. így ezt a jelölést nem használjuk. Vezessük be helyette a n-edik komplex egységgyök fogalmát:
                  \[\varepsilon_k = \textrm{cos}\bigg(\frac{2k\pi}{n}\bigg)+i\cdot\textrm{sin}\bigg(\frac{2k\pi}{n}\bigg), \quad k=0,...,n-1\]
                  Ezek után a $w$ gyökeit a $z$ és az n-edik komplex egységgyökök segítségével kaphatjuk meg: $z\varepsilon_0, ..., z\varepsilon_{n-1}$
        \end{itemize}

\end{description}
\subsection{Leszámlálások véges halmazokon}
\begin{description}
    \item[Véges halmazok] \hfill
        \begin{itemize}
            \item Halmazok ekvivalenciája \\
                  $X,Y$ halmazok ekvivalensek, ha létezik $X$-et $Y$-ra képező bijekció.\\
                  Jele: $X \sim Y$
            \item Véges és végtelen halmazok \\
                  $X$ halmaz véges, ha $\exists n\in\N : X \sim \{1,2,...,n\}$, egyébként végtelen. Ha létezik $n$, akkor az egyértelmű, és ekkor a halmaz elemszámának/számosságának nevezzük. Jele: $ \#(X)$
        \end{itemize}
    \item[Skatulya elv] \hfill \\
        Ha $X,Y$ véges halmazok és $\#(X) > \#(Y)$, akkor egy $f:X\rightarrow Y$ leképezés nem lehet kölcsönösen egyértelmű (azaz bijekció).
    \item[Leszámolások] \hfill
        \begin{itemize}
            \item Permutáció \\
                  $A$ halmaz egy permutációja az önmagára való kölcsönösen egyértelmű leképezése. Az $A$ halmaz összes permutációjának száma:
                  \[P_n = \prod\limits_{k=1}^{n} k \ = \ n!\]
            \item Variáció \\
                  Az $A$ halmaz elemeiből készíthető, különböző tagokból álló $a_1,a_2,...,a_k$ sorozatokat az $A$ halmaz $k$-ad osztályú variációinak nevezzük. Ha $A$ véges ($\#(A) = n$), akkor $V_n^k$ száma megegyezik az $\{1,2,...,k\}$-t $\{1,2,...,n\}$-be képező kölcsönösen egyértelmű leképezések számával:
                  \[ V_n^k = \frac{n!}{(n-k)!}\]
            \item Kombináció \\
                  Ha $A$ halmaz $k\in\N$ elemű részhalmazait $k$-ad osztályú kombinációinak nevezzük. Ha $A$ véges, akkor $C_n^k$ száma megegyezik $\{1,2,...,n\}$ $k$ elemű részhalmazainak számával.
                  \[C_n^k = \dbinom{n}{k} = \frac{n!}{k!(n-k)!}\]
            \item Ismétléses permutáció \\
                  $A = \{a_1,\dotsc,a_r\}$ halmaz elemeinek ismétlődései $i_1,\dotsc,i_r$. (Az elemek ismétléses permutációi olyan $i_1+\cdots+i_r = n$ tagú sorozatok, melyben az $a_j$ elem $i_j$-szer fordul elő.)
                  \[P_n^{i_1,\dotsc,i_r} = \frac{n!}{i_1!i_2!\cdots i_r!}\]
            \item Ismétléses variáció \\
                  Az $A$ véges halmaz elemeiből készíthető (nem feltétlenül különböző)  $a_1,\cdots,a_k$ sorozatokat, az $A$ halmaz $k$-ad osztályú ismétléses variációinak nevezzük.
                  \[^iV_n^k = n^k\]
            \item Ismétléses kombináció \\
                  Az $A$ véges halmaz. A halmazból $k$ elemet kiválasztva, ismétléseket megengedve, de a sorrend figyelmen kívül hagyva, az $A$ halmaz $k$-ad osztályú ismétléses kombinációit kapjuk.
                  \[^iC_n^k = \dbinom{n+k-1}{k} \]
        \end{itemize}
    \item[Tételek] \hfill \\
        \begin{itemize}
            \item Binomiális tétel \\
                  $x,y \in R$  (kommutatív egységelemes gyűrű), $n\in\R$. Ekkor
                  \[(x+y)^n = \sum\limits_{k=0}^n\dbinom{n}{k}x^ky^{n-k} \]
            \item Polinomiális tétel \\
                  $r,n\in\N$ és $x_1, x_2, \cdots, x_r \in R$ (kommutatív egységelemes gyűrű), ekkor
                  \[(x_1+\cdots+x_r)^n = \sum\limits_{i_1+\cdots+i_r = n}P_n^{i_1,\cdots,i_r}x_1^{i_1}x_2^{i_2}\cdots x_r^{i_r} \qquad (i_1,\cdots,i_r \in\N)\]
            \item Szita formula \\
                  $X_1,\cdots,X_k \subset X$ (véges halmaz). $f$ az $X$-en értelmezett, egy Abel-csoportba képző függvény. Legyen:
                  \[S=\sum\limits_{x\in X}f(x)\]
                  \[ S_r = \sum\limits_{1\leq i_1 \leq \cdots \leq i_r \leq k}\Bigg(\sum\limits_{x \in X_{i_1} \cap \cdots \cap X_{i_r}}f(x)\Bigg) \]
                  és
                  \[S_0 = \sum\limits_{x\in X \setminus \cup_{i=1}^k Xi}f(x)\]
                  Ekkor
                  \[S_0 = S - S_1+S_2-S_3+\cdots+(-1)^kS_k \]
        \end{itemize}
\end{description}
\subsection{Számelméleti alapfogalmak, maradékos osztás, lineáris kongruencia-egyenletek}
\subsubsection{Számelméleti alapfogalmak}
\begin{description}
    \item[Oszthatóság egységelemes integritási tartományban] \hfill \\
        $R$ egységelemes integritási tartomány, $a,b\in R$. Ha $\exists c\in R: a = bc$, akkor $b$ osztója  $a$-nak ($a$ a $b$ többszöröse). Jele: $b|a$

        A $b = 0$-t kivéve legfeljebb egy ilyen $c$ létezik.

        Az oszthatóság tulajdonságai egységelemes integritási tartományban.
        \begin{itemize}
            \item Ha $b|a$ és $b'|a'$, akkor $bb'|aa'$
            \item $\forall a\in R: a|0$ (a nullának minden elem osztója)
            \item $0|a \Leftrightarrow a = 0$ (a null csak saját magának osztója)
            \item $\forall a\in R: 1|a$ (az egységelem minden elem osztója)
            \item $b|a \Rightarrow \forall c \in R : bc|ac$
            \item $bc|ac$ és $c\neq0 \Rightarrow b|a$
            \item $b|a_i$ és $c_i \in R,\ (i = 1,\cdots,j) \ \Rightarrow \ b|\sum_{i=1}^{j}a_ic_i$
            \item az $|$ reláció reflexív és tranzitív
        \end{itemize}
    \item[Felbonthatatlan elem és prímelem] \hfill \\
        $0,1 \neq a \in R$ felbonthatatlan (irreducibilis), ha $a = bc$ esetén $b$ vagy $c$ egység ($b,c \in R$).

        $0,1 \neq p \in R$ prím, ha $\forall a,b \in R : p|ab$ esetén $p|a$ vagy $p|b$
    \item[Legnagyobb közös osztó, legkisebb közös többszörös, relatív prím] \hfill \\
        $R$ egységelemes integritási tartomány. $a_1,\cdots,a_n \in R$ elemeknek $b\in R$ legnagyobb közös osztója, ha $b|a_i$ és $b'|a_i$ esetén $b'|b$. Ha $b$ egység, akkor $a_1, \cdots , a_n$ relatív prímek.

        $a_1,\cdots,a_n \in R$ elemeknek legkisebb közös többszöröse $b\in R$, ha $a_i|b$ és $a_i|b'$ esetén $b|b'$.
    \item[Bővített euklideszi algoritmus] \hfill \\
        Az eljárás meghatározza az $a,b \in \Z$ számok legnagyobb közös osztóját ($d\in\Z$), valamint $x,y \in\Z$ számokat úgy, hogy $d = ax+by$
    \item[A számelmélet alaptétele] \hfill \\
        Minden pozitív természetes szám (sorrendtől eltekintve) egyértelműen felbontható prímszámok szorzataként.
    \item[Erathoszthenész szitája] \hfill \\
        Adott $n$-ig a prímek meghatározásához:
        Írjuk fel a számokat 2-től $n$-ig. Az első szám (2) prím, összes többszöröse összetett, ezeket húzzuk ki. A fennmaradó számok közül az első (3) ugyancsak prím, stb.  Az eljárás végén az $n$-nél nem nagyobb prímek maradnak.
\end{description}
\subsubsection{Maradékos osztás}
Legyen $R$ egységelemes integritási tartomány, $f,g\in R[x], g \neq 0$ és tegyük fel, hogy $g$ főegyütthatója egység $R$-ben. Ekkor
\[ \exists! q,r \in R[x] : f = g\cdot q+r \qquad (\text{deg}(r) < \text{deg}(g) )\]

\subsubsection{Horner-séma}
A Horner-módszer egy polinom helyettesítési értékének kiszámítására alkalmas. (Ezzel együtt természetesen az is eldönthető, hogy adott $c$ érték a polinom gyöke-e vagy nem. 4-ed fok felett erre még analitikus megoldás sincs.)

A módszer lényege, hogy az egyébként $f_nx^n+f_{n-1}x^{n-1}+\cdots+f_0$ polinom helyettesítési értékének kiszámolásához rendkívül sok szorzásra és összeadásra lenne szükség. A polinom átalakításával azonban a műveletek számát lecsökkenthetjük. A maradékos osztást alkalmazva:
\[f_nx^n+f_{n-1}x^{n-1}+\cdots+f_0 = (f_nx^{n-1}+f_{n-1}x^{n-2}+\cdots)x+f_0\]
Ezt rekurzívan folytatva a következő alakra jutunk:
\[(((f_nx+f_n-1)x+f_n-2)x+\cdots)x+f_0\]
A helyettesítési érték kiszámítását egy táblázatban könnyebben elvégezhetjük.

\begin{center}
    \begin{tabular}{|c|c|c|c|c|c|}
        \hline     & $f_n$ & $f_{n-1}$      & $f_{n-2}$                 & $\cdots$ & $f_0$  \\
        \hline $c$ & $f_n$ & $f_nc+f_{n-1}$ & $(f_nc+f_{n-1})c+f_{n-2}$ & $\cdots$ & $f(c)$ \\
        \hline
    \end{tabular}
\end{center}
A táblázat kitöltése a következőképp zajlik:
\begin{enumerate}
    \item Az első sorba felírjuk a polinom együtthatóit
    \item A második sor első cellájába beírjuk az argumentum értékét.
    \item A főegyüttható alá beírjuk önmagát.
    \item A második sor celláinak kitöltésével folytatjuk
    \item \label{itm:horner_recursive} Az előző cella elemét megszorozzuk az argumentummal
    \item A szorzathoz adjuk hozzá az aktuális együtthatót
    \item Az összeget írjuk be az aktuális cellába
    \item Folytassuk az \ref{itm:horner_recursive}. ponttal, míg el nem jutunk az utolsó celláig
\end{enumerate}

\noindent
Az utolsó cellába a polinom helyettesítési értéke kerül. (Ha ez nulla, akkor az argumentum a polinom gyöke. )
\subsubsection{Lineáris kongruencia egyenletek}
\begin{description}
    \item[Kongruencia] \hfill \\
        Ha $a,b,m \in\Z$ és $m|(a-b)$, akkor azt mondjuk, hogy $a$ és $b$ kongruensek modulo $m$ (Jele: ${a \equiv b \mod{m}}$).

        A kongruencia ekvivalenciareláció bármely $m$-re. Ha $a\in\Z$ akkor az ekvivalenciaosztály elemei $a+km, k\in\Z$ alakúak.
    \item[Maradékosztályok] \hfill \\
        Az $m\in\Z$ modulus szerinti ekvivalenciaosztályoknak nevezzük. A maradékosztályokat elemeikkel reprezentáljuk. (Az $a$ elem által reprezentált maradékosztály $\widetilde{a} \mod{m}$).

        Ha egy maradékosztály valamely eleme relatív prím a modulushoz, akkor mindegyik az és a maradékosztályt redukált maradékosztálynak nevezzük.

        Páronként inkongruens egészek egy rendszerét maradékrendszernek nevezzük.

        Ha egy maradékrendszer minden maradékosztályból tartalmaz elemet, akkor teljes maradékrendszer.

        Ha maradékrendszer pontosan a redukált maradékosztályokból tartalmaz elemet, akkor redukált maradékrendszer.
    \item[Euler-féle $\varphi$ függvény] \hfill \\
        $m > 0$ egész szám. Az Euler-féle $\varphi(m)$ függvény a modulo $m$ redukált maradékosztályok számát adja meg. Ez nyilván megegyezik a $0,1,\cdots,m-1$ számok közötti, $m$-hez relatív prímek számával.
    \item[Euler-Fermat tétel] \hfill \\
        $m>1$ egész, $a$ relatív prím $m$-hez, ekkor:
        \[a^{\varphi(m)}\equiv 1 \mod{m}\]
    \item[Fermat tétel] \hfill \\
        Legyen $p$ prím, és $a\in\Z: p\nmid a$, ekkor
        \[a^{p-1}\equiv 1 \mod p \]
    \item[Lineáris kongruencia megoldása] \hfill \\
        Keressük az $ax \equiv b \mod{m}$ kongruencia megoldásait ($a,b,m\in\Z$ ismert). Ez ekvivalens azzal, hogy keressünk olyan $x$-et, melyre (valamely $y$-nal) $ax+my = b$.

        Legyen $d$ = lnko($a,m$). Mivel $d$ osztója $ax+my$-nak, $b$-t is osztania kell, különben nincs megoldás. Így $\frac{a}{d}x+\frac{m}{d}y = \frac{b}{d}$. Ekkor $a'x+m'y = 1$. A bővített euklideszi algoritmus segítségével olyan $u,v$ számokat kapunk, melyekkel $a'u+m'v = 1$ (ui.: $a', m'$ relatív prímek). Az egyenletet $b'$-vel beszorozva $a'ub'+m'vb' = b' \Rightarrow x \equiv ub' \mod{m'}$
    \item[Lineáris kongruenciarendszer megoldása] \hfill \\
        Két lineáris kongruencia esetén a megoldások $x \equiv a \mod{m}$ és $x \equiv b \mod{n}$. A közös megoldáshoz $x = a + my = b+ nz \Leftrightarrow my-nz = b-a$ egyenletet kell megoldani. Akkor és csak akkor van megoldás, ha $d$ = lnko($m,n$) osztója $b-a$-nak. Ekkor a megoldás valamely $x_1$ egésszel ${ x \equiv x_1 \mod{\textrm{lkkt}(m,n)}}$ alakban írható. (Több kongruencia esetén az eljárás folytatható.)
    \item[Kínai maradéktétel] \hfill \\
        $1 < m_1,\cdots,m_n \in\N$ páronként relatív prímek, és $c_1,\cdots,c_n \in\Z$. Az $x \equiv c_j \mod{m_j}$ ($j=1,\cdots,n$) kongruenciarendszer megoldható, és bármely két megoldása kongruens  $\mod{m_1m_2\cdots m_n}$
\end{description}
\section{Polinomok és műveleteik}
\begin{description}
    \item[Definíció] \hfill \\
        Legyen $R$ gyűrű. Egy polinomot egy $\sum_{i=0}^{n} f_ix^i$ alakú véges összegnek tekintünk, ahol $n\in\N, f_i \in R$.

        Az $f_n$ tagot a polinom főegyütthatójának nevezzük.
    \item[Műveletek] \hfill \\
        Legyen $R[x]$ az $f = (f_0, f_1, \cdots)$ végtelen sorozatok feletti gyűrű (polinomok gyűrűje), ahol $f_i \in R$. Ekkor az $R[x]$-beli műveletek:
        \begin{itemize}
            \item Összeadás: \\
                  \[f+g = (f_0+g_0, f_1+g_1,\cdots)\qquad (f,g \in R[x]) \]
            \item Szorzás: \\
                  \[f\cdot g = h = (h_0,h_1, \cdots) \qquad (f,g,h \in R[x]) \text{, ahol} \]
                  \[h_k = \sum\limits_{i+j = k}f_ig_j\]
        \end{itemize}

        \textit{Megjegyzés: Ha $R$ kommutatív, akkor $R[x]$ is az. Ha $R$ egységelemes az 1 egységelemmel, akkor $R[x]$ is az az $(1,0,0, \cdots)$ egységelemmel.}
\end{description}
\section{Gráfok}
\subsection{Általános és síkgráfok}
\begin{description}
    \item[Alapfogalmak] \hfill
        \begin{itemize}
            \item Irányítatlan gráf\\
                  Egy irányítatlan gráf a $G = (V,E, \varphi)$ rendezett 3-as, ahol:\\
                  $V$ - a csúcsok halmaza \\
                  $E$ - élek halmaza \\
                  $\varphi$ - illeszkedési reláció ($\varphi \in E \times V$)

                  \textit{Ha $v\in\varphi(e)$, akkor $v$ illeszkedik az $e$ élre. ($v\in V, e\in E$). Egy élnek mindig két vége van}

            \item Él-, és csúcstípusok
                  \begin{itemize}
                      \item Izolált csúcs \\
                            $v\in V$ izolált csúcs, ha $\nexists e \in E: v\in \varphi(e)$
                      \item Párhuzamos él \\
                            $e,e'\in E$ élek párhuzamos élek, ha $\varphi(e) = \varphi(e')$
                      \item Hurokél \\
                            $e\in E$ hurokél, ha $|\varphi(e) | = 1$
                  \end{itemize}
            \item Irányított gráf \\
                  Egy irányítatott gráf a $G = (V,E, \psi)$ rendezett 3-as, ahol:\\
                  $V$ - a csúcsok halmaza \\
                  $E$ - élek halmaza \\
                  $\psi$ - illeszkedési reláció ($\psi \in E \rightarrow V \times V$)

                  \textit{$\psi(e) = (v,v')$, ahol $v$ az $e$ él kezdőpontja, $v'$ a végpontja.}
        \end{itemize}
    \item[Véges, egyszerű gráfok - alapfogalmak] \hfill
        \begin{itemize}
            \item Egyszerű gráf \\
                  $G$ gráf egyszerű, ha nem tartalmaz párhuzamos vagy hurokéleket
            \item Véges gráf
                  $G= (V,E,\varphi)$ gráf véges, ha $V,E$ véges halmazok.
            \item Szomszédság, fok\\
                  Két él szomszédos, ha van közös pontjuk.\\
                  Két csúcs szomszédos, ha van közös élük. \\
                  $v\in V$ szomszédjainak száma a $v$ foka. [Jele: deg($v$) = d($v$)]
            \item $r$-reguláris gráfok\\
                  $G$ gráf $r$-reguláris, ha minden pont foka $r$
            \item Teljes gráf\\
                  $G$ gráf teljes gráf, ha minden él be van húzva, más szóval $(|V|-1)$-reguláris. (Jele: $K_{|V|}$)
            \item Páros gráf\\
                  $G$ páros gráf, ha $V = V' \cup V''$ és $V'\cap V'' = \emptyset$ (diszjunkt), valamint él csak $V'$ és $V''$ között fut.

                  \textit{Ha viszont így $V'$ és $V''$ között minden él be húzva, akkor teljes páros gráf. (Jele: $K_{n,m}$, ahol $n=|V'|, m=|V''|$)}
            \item Részgráf \\
                  $G = (V,E,\varphi)$ részgráfja $G'=(V',G',\varphi')$-nek, ha $V\subset V \ \land \ E \subset E' \ \land \ \varphi \subset \varphi'$
            \item Séta, vonal, út \\
                  $G$ gráfban egy $n$ hosszú séta $v$-ből $v'$-be egy olyan
                  \[v_0,e_1,v_1,\cdots,v_{n-1},e_n,v_n\]
                  sorozat, melyre $v=v_1, v'=v_n$ és $v_{i-1},v_{i} \in \varphi(e_i)$

                  Egy séta vonal, ha minden él legfeljebb egyszer szerepel a sorozatban.

                  Egy vonal út, ha minden csúcs legfeljebb egyszer szerepel a sorozatban.

                  Egy séta/vonal/út zárt, ha kezdő és végpontja megegyezik, egyébként nyílt.
            \item Összefüggő gráf
                  Egy gráf összefüggő, ha bármely két csúcs közt van út.

                  \textit{Ez a reláció ekvivalenciareláció, melynek ekvivalenciaosztályait komponenseknek nevezzük.}
            \item Címkézett, Súlyozott gráf \\
                  $G=V,E,\varphi, C_e, c_e, C_v,c_v)$ rendezett 7-es címkézett gráfot jelöl, ahol $C_e, C_v$ tetszőleges halmazok, és
                  \[c_e:E\rightarrow C_e\]
                  \[c_v:E\rightarrow C_v\]

                  Ha $C_e = C_v = \R^+$, akkor a gráfot súlyozott gráfnak nevezzük, és $w$ a csúcs/él súlya. (${w(e)=c_e(e)},\ {w(v) = c_v(v)}$)
        \end{itemize}
    \item[Síkba rajzolhatóság] \hfill
        \begin{description}
            \item[Fogalmak] \hfill
                \begin{itemize}
                    \item Síkba rajzolhatóság \\
                          Egy gráf síkba rajzolható, ha lerajzolható úgy, hogy az elei nem keresztezik egymást.
                    \item Topologikus izomorfia \\
                          Két gráf topologikusan izomorf, ha a következő lépést illetve fordítottját véges sok ismétlésével egyikből a másikat kapjuk: Egy másodfokú csúcsot elhagyunk, és a szomszédjait összekötjük.
                    \item Tartomány \\
                          Ha $G$ gráf síkba rajzolható, akkor a tartományok az élek által határolt síkidomok. (A nem korlátolt síkidom is tartomány.)
                \end{itemize}
            \item[Tételek] \hfill
                \begin{enumerate}
                    \item Minden véges gráf $\R^3$-ban lerajzolható.
                    \item Ha egy véges gráf síkba rajzolható $\Longleftrightarrow$ gömbre rajzolható
                    \item Euler-tétel: \\
                          Ha a $G$ véges gráf összefüggő, síkba rajzolható gráf, akkor:
                          \[|E|+2 = |V|+|T|\]
                    \item Kuratowsky-tétel: \\
                          Egy véges gráf pontosan akkor síkba rajzolható, ha nem tartalmaz $K_5$-tel, vagy $K_{3,3}$-mal topologikusan izomorf részgráfot.
                \end{enumerate}
        \end{description}
\end{description}
\subsection{Fák}
\begin{description}
    \item[Fa]\hfill \\
    Egy gráfot fának nevezünk, ha összefüggő és körmentes.
    \item[Feszítőfa] \hfill \\
        $F$ részgráfja $G$-nek. Ha $F$ fa és csúcsainak halmaza megegyezik $G$ csúcsainak halmazával, akkor $F$-et a $G$ feszítőfájának nevezzük.
    \item[Tételek] \hfill
        \begin{itemize}
            \item Ha $G$ egyszerű gráf, akkor a következő feltételek ekvivalensek:
                  \begin{enumerate}
                      \item $G$ fa
                      \item $G$ összefüggő, de bármely él törlésével már nem az
                      \item Két különböző csúcs között csak egy út van
                      \item $G$ körmentes, de egy él hozzáadásával már nem az
                  \end{enumerate}
            \item Ha $G$ egyszerű véges gráf, akkor a következő feltételek ekvivalensek:
                  \begin{enumerate}
                      \item $G$ fa
                      \item $G$-ben nincs kör és $n-1$ éle van
                      \item $G$ összefüggő és $n-1$ éle van
                  \end{enumerate}
        \end{itemize}
    \item[Irányított fa] \hfill \\
        Olyan fa, melyre: $\exists v \in V : d^-(v) = 0$ és $\forall v'\neq v : d^-(v') = 1$
        (Egy csúscs befoka 0, a többié 1)

        További fogalmak:
        \begin{itemize}
            \item $r \in V, d^-(r) = 0 $ csúcsot gyökérnek nevezzük
            \item $v'$ csúcs szintje a $r,v'$ út hossza
            \item $(v,v')\in\psi(e)$, a $v$ szülője $v'$-nek, $v'$ gyereke, $v$-nek.
            \item $v$ levél, ha $d^+(v)=0$
        \end{itemize}
\end{description}
\subsection{Euler- és Hamilton-gráfok}
\subsubsection{Euler-gráf}
\begin{description}
    \item[Euler-vonal] \hfill \\
        Az Euler-vonal olyan vonal $v$-ből $v'$-be a gráfban, amelyben minden él szerepel. Ha $v=v'$ akkor ezt a vonalat Euler-körvonalnak  is szokás nevezni. Euler-vonallal rendelkező gráfot Euler-gráfnak nevezik.
    \item[Tétel] \hfill \\
        Egy összefüggő véges gráfban pontosan akkor létezik Euler-körvonal, ha minden csúcs páros fokú.
\end{description}
\subsubsection{Hamilton-gráf}
A Hamilton-út egy olyan út $v$-ből $v'$-be a gráfban, mely minden csúcsot tartalmat. Ha $v=v'$ akkor ezt az utat Hamilton-körnek is szokás nevezni. Hamilton-úttal rendelkező gráfot Hamilton-gráfnak nevezik.
\subsection{Gráfok adatszerkezetei}
Gráfok számítógépes reprezentációjához legtöbbször láncolt listákat, vagy mátrixokat szoktak használni. A láncolt listák inkább ritka gráfokra, míg a mátrixok sűrű gráfok esetén gazdaságosak.
\begin{description}
    \item[Illeszkedési mátrix] \hfill \\
        $G = (V,E, \psi)$ irányított gráf esetén a gráfot egy $A = \{0,1,-1\}^{n\times m}$ mátrix segítségével tudjuk reprezentálni, ahol $V = \{v_1,\cdots, v_n\}$, és $E = \{e_1,\cdots,e_m\}$. Ekkor a mátrix egyes elemei:
        \[ a_{ij} =
            \left\{
            \begin{array}{ll}
                1  & \mbox{ha} \ v_i \ \text{kezdőpontja} \ e_j\text{-nek} \\
                -1 & \mbox{ha} \ v_i \ \text{végpontja} \ e_j\text{-nek}   \\
                0  & \mbox{különben}
            \end{array}
            \right.
        \]

        Ha $G$ nem irányított, akkor $a_{ij} = |a_{i,j}|$
    \item[Csúcsmátrix] \hfill \\
        A  fenti jelölésekkel irányított esetben $B \in \Z^{n\times n}$, ahol $b_{ij}$ a $v_i$-ből $v_j$-be menő élek számát jelöli.

        Ha $G$ irányítatlan, akkor $b_{ii}$ $v_i$ hurokéleinek száma, egyébként $b_{ij}$ a $v_i$ és $v_j$ csúcsok közötti élek száma.
\end{description}

\end{document}