\documentclass[margin=0px]{article}

\usepackage{listings}
\usepackage[utf8]{inputenc}
\usepackage{graphicx}
\usepackage{float}
\usepackage[a4paper, margin=0.7in]{geometry}
\usepackage{amsthm}
\usepackage{amssymb}
\usepackage{t1enc}
\usepackage{fancyhdr}
\usepackage{setspace}

\onehalfspacing

\newenvironment{tetel}[1]{\paragraph{#1 \\}}{}

\pagestyle{fancy}
\lhead{\it{PTI BSc Záróvizsga tételek}}
\rhead{15. Adatszerkezetek és adattípusok}

\title{\textbf{{\Large ELTE IK - Programtervező Informatikus BSc} \vspace{0.2cm} \\ {\huge Záróvizsga tételek}} \vspace{0.3cm} \\ 15. Adatszerkezetek és adattípusok}
\author{}
\date{}

\begin{document}
\maketitle

TODO: Erősen hiányos!

\begin{tetel}{Adatszerkezetek és adattípusok}
    Tömb, verem, sor, láncolt listák; bináris fa, általános fa, bejárások, ábrázolások; bináris kupac, prioritásos sor; bináris kereső fa és műveletei, AVL fa, B+ fa; hasító táblák, hasító függvények, kulcsütközés és feloldásai: láncolással, nyílt címzéssel, próbasorozat; gráfok ábrázolásai.
\end{tetel}

\section{Egyszerű adattípusok ábrázolásai, műveletei és fontosabb alkalmazásai}

\subsection{Adattípus}

\textit{Adatszerkezet}: $\sim$ struktúra. \\
\textit{Adattípus}: adatszerkezet és a hozzá tartozó műveletek. \\
\textit{Adatszerkezetek}:
\begin{itemize}
    \item \textit{Tömb}: azonos típusú elemek sorozata, fix méretű.
    \item \textit{Verem}: Mindig a verem tetejére rakjuk a következő elemet, csak a legfelsőt kérdezhetjük le, és vehetjük ki.
          \begin{figure}[H]
              \centering
              \includegraphics[width=0.3\textwidth]{img/Verembe.jpg}
              \includegraphics[width=0.3\textwidth]{img/Verembol.jpg}
              \caption{Verem műveletei}
          \end{figure}
    \item \textit{Sor}: Egyszerű, elsőbbségi és kétvégű. A prioritásos sornál az elemekhez tartozik egy érték, ami alapján rendezhetjük  őket.
          \begin{figure}[H]
              \centering
              \includegraphics[width=0.3\textwidth]{img/Sorba.jpg}
              \includegraphics[width=0.3\textwidth]{img/Sorbol.jpg}
              \caption{Sor műveletei}
          \end{figure}
    \item \textit{Lista}: Láncolt ábrázolással reprezentáljuk. 3 szempont szerint különböztethetjük meg a listákat: fejelem van/nincs, láncolás iránya egy/kettő, ciklusosság van/nincs. Ha fejelemes a listánk, akkor a fejelem akkor is létezik, ha üres a lista. \\
          A lista node-okból áll, minden node-nak van egy, a következőre mutató pointere, illetve lehet az előzőre is, ha kétirányú. Ezen kívül van egy első és egy aktuális node-ra mutató pointer is, és az utolsó elem mutatója NIL. A listát megvalósíthatjuk úgy, hogy tetszőleges helyre lehessen elemet beszúrni, illetve törölni.
          \begin{figure}[H]
              \centering
              \includegraphics[width=0.3\textwidth]{img/Listaba.jpg}
              \includegraphics[width=0.3\textwidth]{img/Torol.jpg}
              \caption{Lista műveletei}
          \end{figure}
    \item \textit{Fa}: Egyszerű, bináris és speciális (kupac, bináris keresőfa, AVL-fa). A bináris fát rekurzívan definiáljuk: $t \in T(E)$ [bin. fák típusérték halmaza(alaptípus)] $\iff$ $t$ üres fa (jele: $\Omega$), vagy $t$-nek van gyökéreleme, $bal(t)$, $jobb(t)$ részfája. Láncoltan ábrázoljuk, tömbösen csak teljes fák, illetve kupac esetén.
    \item \textit{Kupac}: Olyan bináris fa, melynek alakja majdnem teljes és balra rendezett. Tömbösen ábrázoljuk, mert pointeresen a bonyolult lépkedést nem teszi lehetővé, tömbösen indexösszefüggésekkel könnyen megoldható.
          \begin{figure}[H]
              \centering
              \includegraphics[width=0.3\textwidth]{img/KupacbaBeszur.jpg}
              \caption{Kupac műveletei}
          \end{figure}
    \item \textit{Hasítótábla}
    \item \textit{Gráf} [Nem egyszerű adattípus.]
\end{itemize}

\subsection{Ábrázolásai}

Absztrakciós szintek:
\begin{enumerate}
    \item \textit{absztrakt adattípus (ADT)}: specifikáció szintje, itt nincs szerkezeti összefüggés, csak matematikai fogalmak, műveletek logikai axiómákkal vagy előfeltételekkel.
          \begin{itemize}
              \item algebrai (axiomatikus) specifikáció, példa: $Verembe : V \times E \to V$. Axióma, példa: $Felso(Verembe(v,e)) = e$
              \item funkcionális (elő- és utófeltételes) specifikáció, példa: (elsőbbségi) sor $S(E), s \in S$ egy konkrét sor, $s = \{(e_1, t_1), ..., (e_n, t_n)\}$, $n \geq 0$. Ha $n=0$, akkor a sor üres. \\
                    $\forall i,j \in [1..n]: i \neq j \to t_i \neq t_j$. \\
                    $Sorbol: S \to S \times E$, $\mathcal{D}_{Sorbol} = S \backslash \{ures\}$. Előfeltétel: $Q = (s = s' \wedge s' \neq \emptyset)$, utófeltétel: $R = (s = s' \backslash \{(e,t)\} \wedge (e,t) \in s' \wedge \forall i (e_i, t_i) \in s' : t \leq t_i)$.
          \end{itemize}
    \item \textit{absztrakt adatszerkezet (ADS)}: kognitív pszichológia szintje, ábrák. Az alapvető szerkezeti tulajdonságokat tartalmazza (nem mindet). Ennek a szintnek is része a műveletek halmaza. Példák: az ábra egy irányított gráf, művelet magyarázata, adatszerkezet definiálása.
          \begin{figure}[H]
              \centering
              \includegraphics[width=1\textwidth]{img/ads.png}
              \caption{ADS}
          \end{figure}
    \item \textit{ábrázolás/reprezentáció}: döntések (tömbös vagy pointeres ábrázolás), a nyitva hagyott szerkezeti kérdések. Egy adatszerkezetet többféle reprezentációval is meg lehet valósítani (pl. prioritásos sor lehet rendezetlen tömb, rendezett tömb, kupac).
          \begin{itemize}
              \item \textit{tömbös ábrázolás}: takarékos ábrázolás, elhelyezése, tetszőleges rákövetkezések, bejárások, de ezeket meg kell adni.
              \item \textit{pointeres ábrázolás}: minden pointer egy összetett rekord elejére mutat.
          \end{itemize}
    \item \textit{implementálás}
    \item \textit{fizikai szint}
\end{enumerate}

\subsection{Műveletei}

\begin{itemize}
    \item Üres adatszerkezet létrehozása
    \item Annak lekérdezése, hogy üres-e az adatszerkezet
    \item Elem berakása, itt ellenőrizni kell, hogy nem telt-e még meg
    \item Elem kivétele vagy törlése, itt ellenőrizni kell, hogy nem üres-e
    \item Adott tulajdonságú elem (például maximum, veremben a felső) lekérdezése, itt is ellenőrizni kell, hogy üres-e az adatszerkezet
    \item Bejárások (preorder, inorder, postorder, szintfolytonos), listáknál az első, előző vagy következő elemre lépés
    \item Elem módosítása bizonyos adatszerkezeteknél (pl. listák)
\end{itemize}

\subsection{Fontosabb alkalmazásai}

\textit{Prioritásos sor}: nagygépes programfuttatásnál az erőforrásokat a prioritás arányában osszuk el, adott pillanatban a maximális prioritásút válasszuk. Sürgősségi ügyeleten, gráfalgoritmusoknál is alkalmazható.
\textit{B-fa}: ipari méretekben adatbázisokban használják.

\section{A hatékony adattárolás és visszakeresés néhány megvalósítása (bináris keresőfa, AVL-fa, 2-3-fa és B-fa, hasítás (,,hash-elés”))}

\subsection{Bináris keresőfa}

Nincsenek benne azonos kulcsok, a követendő elv: ,,kisebb balra, nagyobb jobbra". Inorder bejárással növekvő kulcssorozatot kapunk. \\
Műveletigénye fa magasságú nagyságrendű. \\
Az a cél, hogy a bináris keresőfa ne nyúljon meg láncszerűen, erre jó az AVL-fa és a 2-3-fa.
\begin{figure}[H]
    \centering
    \includegraphics[width=0.3\textwidth]{img/Beszur.jpg}
    \includegraphics[width=0.3\textwidth]{img/Kovetkezo.jpg}
    \caption{Bináris keresőfa műveletei}
\end{figure}

\subsection{AVL-fa}

Cél: a $t$ bináris keresőfa magasságának $\log_2(n)$ közelében tartása, azaz $h(t) \leq c \cdot \log_2(n)$, ahol $c$ elfogadhatóan kicsi. Az ilyen fát kiegyensúlyozottnak nevezzük. \\
\textit{AVL}: Adelszon-Velszkij, Landisz 1962-ben alkották meg. \\
A $t$ bináris keresőfát egyúttal AVL-fának nevezzük $\iff$ $t$ minden $x$ csúcsára $|h(bal(x))-h(jobb(x))| \leq 1$. \\
Minden csúcsnak van egy címkéje $+,-,=$ (gyerekek magasságának különbsége). A beszúrás helyétől felfelé ellenőrizzük ezeket, és ha kell, akkor módosítjuk. Ha valahol $++$ vagy $--$ alakul ki, akkor ott elromlik az AVL-tulajdonság, egy vagy több forgatással vagy átkötéssel konstans műveletigénnyel helyre lehet hozni. \\
Többféle séma is van: $(++,+), (++,-), (++,=)$ és a tükörképeik.

\subsection{2-3-fa és B-fa}

2-3-fa kis méretben az elmélet számára jó, a B-fa a gyakorlati változat adatbázisban. \\
$t$ 2-3-fa $\iff$ minden belső csúcsnak 2 vagy 3 gyereke van, a levelek azonos szinten helyezkednek el, adatrekordok csak a levelekben vannak, belső pontokban kulcsok és mutatók, levelekben a kulcsok balról jobbra nőnek. \\
Ha 4 gyerek lenne a beszúrás után, akkor csúcsot kell vágni. Ha törlésnél 1 gyerek lenne valahol, akkor csúcsösszevonásokat és gyerekátadást alkalmazunk. \\
B-fa nagyobb méretű, itt két határ között mozog a gyerekszám: $\lceil\frac{r}{2}\rceil$ és $r$, ahol $50 \leq r \leq 1000$.

\subsection{Hasítás}

Kulcsos rekordokat tárol.
\begin{itemize}
    \item \textit{Hasítás láncolással}: a kulcsütközést láncolással oldja fel. Van egy hasítófüggvény: $h: U \to [0..m-1]$, elvárás vele kapcsolatban, hogy gyorsan számolható és egyenletes legyen. $m$-et úgy választjuk meg $n$ nagyságrendjének ismeretében, hogy $\alpha = \frac{n}{m}$ lesz a várható listahossz, ha egyenletes hasítást feltételezünk.\\
          Például kétirányú listát használhatunk a hasításhoz. Műveletek: beszúrás, keresés, törlés. \\
          Gyakorlatban érdemes $m$-et úgy megválasztani, hogy olyan prímszám legyen, ami nem esik 2-hatvány közelébe.
    \item \textit{Hasítás nyitott/nyílt címzéssel}: A kulcsokat lehessen egészként értelmezni, ekkor vannak jó hasítófüggvények. \\
          Próbálkozás általános képlete: $h(k) + h_i(k)$ $(mod$ $M)$, $0 \leq i \leq M-1$. Egész addig alkalmazza, amíg üres helyet nem talál.
          \begin{enumerate}
              \item \textit{Lineáris próba}: $h_i(k) = -i$ $(mod$ $M)$, egyesével balra lépegetve keressük az üres helyet. Hátránya az elsődleges csomósodás, ez jelentős lassulást okoz beszúrásnál és keresésnél.
              \item \textit{Négyzetes próba}: $h_i(k) = (-1)^i(\lceil\frac{i}{2}\rceil)^2$ $(mod$ $M)$, a négyzetszámokkal lépegetünk balra-jobbra, ezek az eltolások kiadják $\{0,1,...,M-1\}$-et. Hátrány: másodlagos csomósodás.
              \item \textit{Kettős hash-elés}: $h_i(k) =-ih'(k)$ $(mod$ $M)$, $h'(k)$ a $k$-hoz tartozó egyedi lépésköz, $(h'(k),M)=1$ relatív prímek. Ha az $M$ elég nagy, akkor nincs csomósodás.
          \end{enumerate}
    \item \textit{Hasítófüggvények}: Leggyakoribb: $k$ egész, kongruencia reláció. Általánosan: $h(k) = (ak +b$ $(mod$ $p))$ $(mod$ $M)$, az univerzális hasítás családja. Tapasztalat: $k$ egyenletesen hasít.
\end{itemize}

\end{document}